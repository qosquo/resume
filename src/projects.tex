%-----------PROJECTS-----------%
\section{Проекты}
\resumeSubHeadingListStart

    \resumeProjectHeading
	{\textbf{{\href{https://github.com/qosquo/trachtenberg}{Классификация коротких текстовых данных (NLP) \faGithub}}} $|$ \emph{Scikit-Learn, FastAPI, NextJS, Docker}}{}
    \resumeItemListStart
		\resumeItem{Разработал бинарный классификатор для определения авторского стиля в текстах: сбор текстов (парсинг), очистка данных (обработка, орфография), векторизация через TF-IDF и FastText.}
        \resumeItem{Обучил и сравнил 2 модели (Logistic Regression, XGBoost).}
        \resumeItem{Модель показала F1-score 0.37, демонстрируя сложность автоматической классификации субъективного юмора и предоставив практический опыт работы с неравномерно размеченными данными}
		\resumeItem{Создал \underline{\href{http://arena.rrzagitov.xyz}{веб-интерфейс}} для демонстрации модели (Next.js), что позволило тестировать её на реальных примерах.}
    \resumeItemListEnd

    \resumeProjectHeading
	{\textbf{{\href{https://github.com/risen09/eng-it-lean}{Микрофронтенд-приложение для изучения английского языка \faGithub}}} (учебный проект) $|$ \emph{React, Node.js, Redux, Jest}}{}
    \resumeItemListStart
		\resumeItem{Разрабатывал в команде из трёх человек в рамках учебного курса Технохаба Сбера.}
        \resumeItem{Оптимизировал код, переведя вёрстку с HTML на JSX}
		\resumeItem{Реализовал REST API на Node.js для взаимодействия между клиентом и сервером. Подключил Redux для управления состоянием приложения.}
		\resumeItem{Интегрировал GigaChat API через библиотеку \emph{vercel/ai} как генератор уроков английского языка.}
		\resumeItem{Приложение доступно на \underline{\href{https://dev.bro-js.ru/eng-it-lean}{dev.bro-js.ru/eng-it-lean}}.}
    \resumeItemListEnd

    \resumeProjectHeading
	{\textbf{{\href{https://github.com/qosquo/uni-ml/blob/master/uni-cl-alg/uni-cl-alg-youtube.ipynb}{Анализ истории YouTube-просмотров}} \faGithub} (учебный проект) $|$ \emph{Pandas, Matplotlib}}{}
    \resumeItemListStart
		\resumeItem{Исследовал собственные данные через YouTube Takeout: парсинг JSON, очистка данных, визуализация (Matplotlib).}
        \resumeItem{Изучил продвинутые методы Pandas через перевод материалов книги \href{https://link.springer.com/book/10.1007/979-8-8688-0413-7}{\emph{<<Numerical Python>>} (автор: Robert Johansson)}.}
		\resumeItem{Из анализа получил, что 92\% просмотренных мною видео длятся менее 20 минут. Мне нравятся примерно 8\% просмотренных видео.}
    \resumeItemListEnd

\resumeSubHeadingListEnd
