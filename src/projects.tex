%-----------PROJECTS-----------%
\section{Проекты}
\resumeSubHeadingListStart

    \resumeProjectHeading
	{\textbf{Индивидиум -- ИИ-репититор} (\underline{\href{https://github.com/risen09/ReactExpo}{фронтенд}}, \underline{\href{https://github.com/risen09/vpsnode}{бэкенд}} \faGithub) $|$ \emph{React Native, Node.js, Express.js, Langchain.js}}{}
    \resumeItemListStart
		\resumeItem{Разработал мобильное приложение в команде из 3 человек: персонализированный ИИ-репетитор, адаптирующий школьную программу под ученика.}
		\resumeItem{Запустил продукт: подготовил Android-версию (APK) и обеспечил стабильную работу на реальных устройствах. Проект также подан на стартап-программу.}
		\resumeItem{Спроектировал и реализовал 3 ИИ-агента на основе GigaChat -- они подстраивают объяснения под уровень и тип личности ученика.}
        \resumeItem{Упростил вход в приложение через VK ID: повысил доступность для школьников.}
        \resumeItem{Оптимизировал работу с API и логикой агентов, что улучшило стабильность приложения.}
    \resumeItemListEnd

    \resumeProjectHeading
	{\textbf{{\href{https://github.com/qosquo/trachtenberg}{Классификация коротких текстов \faGithub}}} $|$ \emph{Scikit-Learn, FastAPI, NextJS, Docker}}{}
    \resumeItemListStart
		\resumeItem{Разработал сервис, определяющий авторский стиль в коротких текстах на основе машинного обучения.}
        \resumeItem{Провёл полный цикл: от сбора и подготовки данных до обучения моделей и развёртывания.}
        \resumeItem{Построенные модели показывали умеренное качество (F1-score до 0.37) -- задача оказалась сложной из-за шумных и несбалансированных данных.}
		\resumeItem{Создал \underline{\href{http://arena.rrzagitov.xyz}{веб-интерфейс}} для демонстрации и тестирования модели в реальном времени.}
    \resumeItemListEnd

    \resumeProjectHeading
	{\textbf{{\href{https://github.com/risen09/eng-it-lean}{Микрофронтенд-приложение для изучения английского языка \faGithub}}} $|$ \emph{React, Node.js, Redux, Jest}}{}
    \resumeItemListStart
		\resumeItem{Участвовал в командной разработке веб-приложения в рамках курса от Сбера (3 человека).}
		\resumeItem{Реализовал REST API (Express.js) и настройку Redux для управления состоянием.}
		\resumeItem{Интегрировал GigaChat как генератор заданий по английскому языку.}
		\resumeItem{Приложение доступно онлайн: \underline{\href{https://dev.bro-js.ru/eng-it-lean}{dev.bro-js.ru/eng-it-lean}}.}
    \resumeItemListEnd

    \resumeProjectHeading
	{\textbf{{\href{https://github.com/qosquo/uni-ml/blob/master/uni-cl-alg/uni-cl-alg-youtube.ipynb}{Анализ истории YouTube-просмотров}} \faGithub} (учебный проект) $|$ \emph{Pandas, Matplotlib}}{}
    \resumeItemListStart
		\resumeItem{Исследовал собственные данные через YouTube Takeout: парсинг JSON, очистка данных, визуализация (Matplotlib).}
        \resumeItem{Изучил продвинутые методы Pandas через перевод материалов книги \href{https://link.springer.com/book/10.1007/979-8-8688-0413-7}{\emph{<<Numerical Python>>} (автор: Robert Johansson)}.}
		\resumeItem{Из анализа получил, что 92\% просмотренных мною видео длятся менее 20 минут. Мне нравятся примерно 8\% просмотренных видео.}
    \resumeItemListEnd

\resumeSubHeadingListEnd
